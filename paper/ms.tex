\documentclass[twocolumn]{aastex62}

% \submitjournal{ApJ}

\shortauthors{Beane et al.}
\shorttitle{The Galactic Midplane is Not a Plane}

\usepackage{graphicx}
\usepackage{gensymb}
\usepackage{bm}
\usepackage{mathtools}
\usepackage{makecell}
\usepackage{scrextend}

\newcommand{\Gus}[1]{\textcolor{red}{#1}}
\newcommand{\MM}[1]{\textcolor{blue}{#1}}   %mm  Mordecai

\newcommand{\Msun}{\ensuremath{\text{M}_\odot}}
\newcommand{\pc}{\text{pc}}
\newcommand{\kpc}{\text{kpc}}
\newcommand{\Myr}{\text{Myr}}
\newcommand{\Gyr}{\text{Gyr}}
\newcommand{\kms}{\text{km}\,\text{s}^{-1}}
\newcommand{\actunit}{\text{kpc}\,\kms}

\newcommand{\unit}[2]{\ensuremath{\textrm{#1}^{\mathrm{#2}}}}

\newcommand{\mi}{\texttt{m12i}}
\newcommand{\mf}{\texttt{m12f}}
\newcommand{\mm}{\texttt{m12m}}

\newcommand{\sgra}{Sgr~A\textsuperscript{*}}

\newcommand{\abs}[1]{\left| #1 \right|}
\newcommand{\avg}[1]{\left< #1 \right>}
\newcommand{\z}{z_r}
\newcommand{\n}{\text{n}}

\newcommand{\beq}{\begin{equation}}
\newcommand{\eeq}{\end{equation}}

\newcommand{\thin}{\texttt{thin-disk}}
\newcommand{\thick}{\texttt{thick-disk}}
\newcommand{\halo}{\texttt{halo}}
\newcommand{\thincolor}{pink}
\newcommand{\thickcolor}{brown}
\newcommand{\halocolor}{blue}

% affiliations
\newcommand{\cca}{Center for Computational Astrophysics, Flatiron Institute,
162 5th Ave., New York, NY 10010, USA}
\newcommand{\penn}{Department of Physics \& Astronomy, University of
Pennsylvania, 209 South 33rd St., Philadelphia, PA 19104, USA}
\newcommand{\amnh}{Department of Astrophysics, American Museum of Natural
History, Central Park West at 79th St., New York, NY 10024, USA}
\newcommand{\columbia}{Department of Astronomy, Columbia University, 550 W
120th St., New York, NY 10027, USA}
\newcommand{\victoria}{Department of Physics \& Astronomy, University of
Victoria, 3800 Finnerty Rd., Victoria, BC, V8P 4HN, Canada}
\newcommand{\nyuphys}{Center for Cosmology and Particle Physics, Department of
Physics, New York University, 726 Broadway, New York, NY 10003, USA}
\newcommand{\nyucds}{Center for Data Science, New York University, 60 Fifth
Ave., New York, NY 10011, USA}
\newcommand{\mpia}{Max-Planck-Institut f\"ur Astronomie, K\"onigstuhl 17,
69117 Heidelberg, Germany}

\begin{document}

\title{The Galactic Midplane Is Not a Plane: Implications for Dynamical
Analysis with \textit{Gaia} Data and Beyond}

\correspondingauthor{Angus Beane}
\email{abeane@sas.upenn.edu}

\author[0000-0002-8658-1453]{Angus Beane}
\affil{\cca}
\affil{\penn}

\author[0000-0003-3939-3297]{Robyn E. Sanderson}
\affil{\penn}
\affil{\cca}

\author{Melissa K. Ness}
\affil{\columbia}
\affil{\cca}

\author{Kathryn V. Johnston}
\affil{\columbia}
\affil{\cca}

\author[0000-0001-8275-9181]{Douglas Grion Filho}
\affil{\columbia}

\author[0000-0003-0064-4060]{Mordecai-Mark Mac Low}
\affil{\amnh}
\affil{\cca}

\author{Daniel Angl\'es-Alc\'azar}
\affil{\cca}

\author[0000-0003-2866-9403]{David W. Hogg}
\affil{\nyuphys}
\affil{\nyucds}
\affil{\cca}
\affil{\mpia}

\author{Chervin F. P. Laporte}
\altaffiliation{CITA National Fellow}
\affil{\victoria}

\begin{abstract}

% Orbital properties of stars, computed from their six-dimensional phase space measurements and an assumed Galactic potential, are used to understand the structure and evolution of the Galaxy. Stellar actions, computed from orbits, are invoked as invariant and thus quantitative labels of a star’s orbit.

%   Quantitatively, an offset in the vertical position of ∼ 15 pc (∼ 120 pc) for a thin (thick) disk orbit is sufficient to induce a 25\% systematic error in the vertical action Jz. Such an offset can be trivially generated by local midplane variations. In the FIRE simulations these are on the order of ~100pc at the solar circle. For the Milky Way, from the mean vertical velocity variation of the Milky Way disk of ∼ 5–10 km/s with radius, combined with the dust-map density distribution, we estimate that the Milky Way midplane variations are of order ∼ 60–170 pc. Action calculations which assume the global and local midplanes are identical, are likely to include the induced errors we report, depending on the volume considered. Variation in the local standard of rest causes similar issues. The variation of the midplane must be taken into account when performing dynamical analysis across the large regions of the disk accessible to Gaia and future missions. 

Orbital properties of stars, computed from their six-dimensional phase space
measurements and an assumed Galactic potential, are used to understand the
structure and evolution of the Galaxy. Stellar actions, computed from orbits,
have the attractive quality of being invariant under certain assumptions and
are therefore used as quantitative labels of a star's orbit. We report a
subtle but important systematic error that is induced in the actions as a
consequence of local midplane variations expected for the Milky Way. This
error is difficult to model because it is non-Gaussian and bimodal, with
neither mode peaking on the null value. An offset in the vertical position of
the Galactic midplane of $\sim15\,\pc$ for a thin disk-like orbit or $\sim
120\,\pc$ for a thick disk-like orbit is sufficient to induce a $25\%$
systematic error in the vertical action $J_z$. In FIRE simulations of Milky
Way-mass galaxies, these variations are on the order of $\sim100\,\pc$ at the
solar circle. From observations of the mean vertical velocity variation of
$\sim5\textup{--}10\,\kms$ with radius, we estimate that the Milky Way
midplane variations are of order $\sim60\textup{--}170\,\pc$\,--\,also
consistent with recent three-dimensional dust maps. Action calculations and
orbit integrations which assume the global and local midplanes are identical
are likely to include the induced errors we report, depending on the volume
considered. Variation in the local standard of rest or distance to the
Galactic center causes similar issues. The variation of the midplane must be
taken into account when performing dynamical analysis across the large regions
of the disk accessible to \textit{Gaia} and future missions.


%  Furthermore, we
% show that the local midplane varies by $\sim100\,\pc$ at the solar circle in
% Milky Way-mass cosmological zoom-in simulations from the FIRE project. From
% observations that the mean vertical velocity of the Milky Way disk varies by
% $\sim5\textup{--}10\,\kms$ with radius, we estimate that the true midplane
% variations are of order $\sim60\textup{--}170\,\pc$, with similar offsets seen
% in recent three-dimensional dust maps. Thus, current state-of-the-art action
% calculations, which assume the global and local midplanes are identical, are
% likely to include a systematic vertical offset depending on the volume
% considered. Variation in the local standard of rest induces similar issues. The
% variation of the midplane must be taken into account when performing dynamical
% analysis across the large regions of the disk accessible to \textit{Gaia} and
% future missions.

\end{abstract}

\keywords{Galaxy: disk -- Galaxy: evolution -- Galaxy: kinematics and dynamics
-- Galaxy: structure -- stars: kinematics and dynamics}

\section{Introduction} \label{sec:intro} Our understanding of the Milky Way is
currently undergoing a revolution as a result of \textit{Gaia} Data Release 2
(DR2). Recent major discoveries include the affirmation of remnants of a major
merger \citep{2018ApJ...860L..11K, 2018MNRAS.478..611B, 2018Natur.563...85H,
2019MNRAS.486..378L, 2019MNRAS.482.3426M} hinted at in pre-\textit{Gaia} work
\citep[e.g.,][]{2005MNRAS.359...93M, 2011MNRAS.412.1203N}, a phase-space
``spiral'' in the solar neighborhood \citep{2018Natur.561..360A} possibly
indicating local substructure infall \citep{2018MNRAS.481.1501B,
2019MNRAS.485.3134L}, and a gap suggestive of perturbation by a dark matter
subtructure in the tidal stream GD1 \citep{2018ApJ...863L..20P,
2018arXiv181103631B}. These discoveries all indicate that the Milky Way's
stellar distribution, which demonstrably departs significantly from axisymmetry,
is undergoing phase mixing and dynamical interactions across a range of spatial
and temporal scales.

The assumption of a global, axisymmetric Galactocentric coordinate system
\citep{2008gady.book.....B} underlies much of the quantitative analysis of the
mechanisms that give rise to these 
    signatures. 
In order to 
construct such a
system, the Sun's relative position and velocity must be defined and measured both precisely
and accurately. This involves determining the angular position of and distance
to the Galactic center, the orientation of and distance to the Galactic 
   midplane,
and the local standard of rest (LSR). We review
and discuss the observational efforts to measure these parameters in
Section~\ref{ssec:coord_off}.

Once a Galactocentric coordinate system has been established and a
six-dimensional (6D) phase space measurement of a star has been made, it is
often desirable to convert this measurement into action space to concisely
summarize its projected orbit, model the stellar distribution function, or find
stars with similar dynamical properties. Actions are conserved quantities that
describe the orbit of a star under the assumption of regular, bound orbits in a
system where the equations of motion are separable in a particular coordinate
system. They are the cyclical integral of the canonical momentum over its
conjugate position: \beq\label{eq:actions} J_i \equiv \frac{1}{2\pi}
\oint_{\text{orbit}}p_i\,\text{d}x_i\text{,} \eeq where $p_i$ are the conjugate
momenta. Under the assumption of axisymmetry, $i=R,z,\phi$ are the radial,
vertical, and azimuthal coordinates respectively in a cylindrical coordinate
system. In a slowly-evolving axisymmetric potential, these actions are invariant
and $J_{\phi} \equiv L_z$, where $L_z$ is the $z$-component of the angular
momentum \citep{2008gady.book.....B,2014RvMP...86....1S}.

With the advent of 6D phase-space measurements over a relatively large ($\gtrsim
2$ kpc) volume from the \textit{Gaia} satellite, the study of stellar actions
has gained new popularity. One reason is dimensionality reduction\,---\,an
individual stellar orbit is concisely described by three actions, as opposed to
six phase space coordinates. Second, under the assumption of a phase-mixed
system, the dynamical properties of a population of stars should be uniquely a
function of their actions and independent of the conjugate angles. This allows
one to use actions to study the relationship between \emph{orbital} properties
of stars and other intrinsic, and, at least partially, invariant properties such
as age or metallicity \citep{2018ApJ...867...31B, 2018arXiv180803278T,
2018MNRAS.481.4093S, 2019arXiv190304030G, 2019arXiv190309320D,
2019MNRAS.486.1167B}. Actions also provide a convenient basis for constructing
models of the stellar distribution function \citep[e.g.,][]{1915MNRAS..76...70J,
1985ApJ...295..388V, 2017ApJ...839...61T}, or for associating stars with similar
dynamical properties, e.g., to potentially determine membership in moving
groups.

If the system being considered departs from axisymmetry in a significant and/or
non-adiabatic way, the actions computed using an axisymmetric approximation to
the true potential can exhibit cyclic dependence on the orbital phase (or time
at which the star's position and velocity are observed), large-scale migration,
or diffusion from their initial values. In the Milky Way, stellar actions are
expected to diffuse on short time scales due to scattering with gas clouds and
to evolve on longer time scales in the case of orbits near resonances with
spiral arms, bar(s), and other large scale perturbations
\citep{2014RvMP...86....1S}. For this reason, actions have been used to study
stellar scattering in the Milky Way disk using the improved astrometry of
\textit{Gaia} DR2 and various age catalogues \citep{2018ApJ...867...31B,
2018arXiv180803278T}. Actions have also been used to study different models of
spiral structure in the Milky Way \citep{2019MNRAS.tmp..155S}. Characteristics
of the distribution of stars in the extended solar neighborhood in action space
are discussed in \citet{2019MNRAS.484.3291T}.

The true Galactic potential is not strictly axisymmetric. This has been known
for some time \citep[e.g.,][]{1957AJ.....62...93K,2009MNRAS.396L..56M,
2012ApJ...750L..41W}, but with the vast improvement in the quality of
phase-space measurements due to \textit{Gaia} 
     the assumption of axisymmetry
is increasingly inadequate \citep[e.g.,][]{2018Natur.561..360A, 2019MNRAS.485.3134L}. Even if
this assumption were close enough for many purposes, the parameters used in
axisymmetric models of the Galactic potential may be inadequately constrained by
current observations.

      Small-scale variations in the density of gas and stars cause the
      local midplane position to vary as a
function of radius and azimuth. 
    Stars far from the Sun have a local midplane that differs
from our local midplane extrapolated onto their position. Converting
positions and velocities of more distant stars from a 
heliocentric to a Galactocentric coordinate system thus introduces a systematic
bias in the $z$ coordinate. We show that this bias induces non-Gaussian errors
in the actions computed for these stars. The further from the solar neighborhood
the target star is, the more likely the mismatch will result in large systematic
uncertainty, especially in the vertical action $J_z$. A similar argument applies
to any remaining uncertainty in measurements of the Galactocentric radius of the
Sun and to variations in the LSR.

\begin{figure*}[ht!]
\begin{center}
\includegraphics[width=\textwidth]{fig/cartoon.pdf}
\end{center}
\caption{Cartoon approximation showing the effect an error in the determination
of the coordinate midplane can have on orbit integration and action estimation.
The $x$-axis shows the orbital phase and the $y$-axis the vertical height. The
top gray curve depicts an example ``true'' orbit oscillating about the true
midplane (horizontal solid gray line). Consider an observer who erroneously
assumes the midplane is located at the horizontal dashed line. Suppose this
observer measures the phase-space position of the star at two different orbital
phases (teal and orange points). If the observer were to then integrate the
star's orbit using a model potential with the erroneous midplane, they would
obtain the teal and orange curves for the star's orbit, respectively. The
actions estimated from these two erroneous measurements would subsequently
differ, both from each other and from the true measurement (in the potential
with the correct coordinate system). Hence an incorrect midplane in the
potential model assumed will induce phase-dependence in the actions estimated
for a given star in that potential.} \label{fig:cartoon} \end{figure*}

In Section~\ref{sec:ref_frame}, we describe the general impact coordinate system
errors have on the measured actions. In Section~\ref{sec:local_fire} we examine
the azimuthal variations of the midplane itself in examples from two classes of
simulations: cosmological, hydrodynamical, zoom-in simulations of isolated Milky
Way-mass galaxies from the Feedback in Realistic Environments (FIRE)
collaboration\footnote{\url{https://fire.northwestern.edu}}
\citep{2014MNRAS.445..581H, 2016ApJ...827L..23W, 2018MNRAS.480..800H}, and a
controlled N-body simulation of a Sagittarius encounter with a galaxy otherwise
tailored to the stellar mass, scale length, and scale height of the Milky Way
\citep{2018MNRAS.481..286L}. In Section~\ref{sec:discussion}, we discuss the
implications of midplane variations, and the resulting systematic uncertainty in
the vertical action, for action-space analyses. We also estimate the expected
midplane variations of the Milky Way based on the observed velocity variations
and three-dimensional dust maps. We summarize our main results and conclude in
Section~\ref{sec:conclusion}.

\section{Motivation} \label{sec:ref_frame}
We first demonstrate the significance to action computations of a systematic
offset in the determination of the Galactic midplane, distance to the Galactic
center, or LSR. We will find that such offsets are especially important for
disk-like orbits. The consequences we explore here may also arise from various
other systematic errors. For instance, the axisymmetric Galactic potential model
used in many works to compute actions may not be a good description of the true
potential --- or the parameters used may yield a potential that is
systematically incorrect outside an original fitted region. In this work, we
assume that the Galaxy is perfectly described by our model axisymmetric
potential, and simply explore the consequences of offsets in the Galactocentric
coordinate system.

\subsection{Effect of Midplane Offset} \label{ssec:cartoon}
We present a cartoon approximation of an orbit in Figure~\ref{fig:cartoon} to
show how an inaccurate or erroneous determination of the midplane leads to a
   dependence on orbital phase of the value of the actions calculated
   from a point in phase space and an
assumed potential model. The $x$-axis corresponds to orbital phase and the
$y$-axis to vertical height. The solid gray curve indicates the 
     true orbit of the star as it oscillates around the true midplane.

The dashed gray line, offset from the true midplane, is the midplane location
used by an observer to integrate the orbit of the star and estimate its actions.
The model potential is otherwise identical to the one in which the star is
actually moving.

Now suppose this observer makes a measurement of the star's position and
velocity at the teal point or the orange point (i.e. at two different orbital
phases). Then the teal and orange curves correspond to the orbits that the
observer would compute for each point based on the potential model with the
offset midplane. In action space, this would correspond to a different value of
$J_z$ for the teal and orange points. In this way, assuming the wrong coordinate
system induces a phase-dependence in the actions estimated for the star, which
in the correct potential (in this example, the one with the correct midplane)
should be phase-independent.

This example uses an offset in $z$, but 
    analagous effects occur from offsets in other coordinates, 
such as the distance to the Galactic
center or the LSR. A similar effect in which actions gain time-dependence due to
a time-varying potential was pointed out by \citet{2015A&A...584A.120B}.

\subsection{Epicyclic Approximation} \label{ssec:epi_action}

\begin{deluxetable*}{cCCCCCCCCC}

\tablecaption{Description and names of the three orbits considered in this work.
For the last four columns: $z_{\text{max}}$ is the maximum height of the orbit,
$\frac{1}{2}(R_{\text{max}} - R_{\text{min}})$ is the magnitude of the radial
excursions of the orbit, and $\kappa$ and $\nu$ are the radial/epicyclic and
vertical frequencies of the orbit. In the epicyclic approximation, $A_z =
z_{\text{max}}$ and $A_R = \frac{1}{2}(R_{\text{max}} - R_{\text{min}})$.
\label{tab:orbits}} \tablehead{\colhead{name} & \colhead{\makecell{initial \\ position}} &
\colhead{\makecell{initial \\ velocity}} & \colhead{$J_R$} & \colhead{$J_{\phi}$} &
\colhead{$J_z$} & \colhead{$z_{\text{max}}$} &
\colhead{$\frac{1}{2}(R_{\text{max}} - R_{\text{min}})$} & \colhead{$\kappa$} &
\colhead{$\nu$} \\ \colhead{ } & \colhead{$(\mathrm{kpc})$} &
\colhead{$(\mathrm{km\,s}^{-1})$} & \colhead{$(\mathrm{kpc\,km\,s^{-1}})$} &
\colhead{$(\mathrm{kpc\,km\,s^{-1}})$} & \colhead{$(\mathrm{kpc\,km\,s^{-1}})$} &
\colhead{$(\mathrm{kpc})$} & \colhead{$(\mathrm{kpc})$} &
\colhead{$(\mathrm{Myr}^{-1})$} & \colhead{$(\mathrm{Myr}^{-1})$}}
\startdata
\thin{} & (8, 0, 0) & (0, -190, 10) & 40 & -1500 & 0.69 & 0.12 & 1.3 & 0.049
& 0.093 \\
\thick{} & (8, 0, 0) & (0, -190, 50) & 33 & -1500 & 23 & 0.85 &
1.2 & 0.048 & 0.061 \\
\halo{} & (8, 0, 0) & (0, -190, 190) & 33 & -1500 &
530 & 6.2 & 2.3 & 0.033 & 0.025
\enddata
\end{deluxetable*}

Before turning to numerical methods, we derive analytic expressions for the
systematic error in the actions induced from offsets in the position (e.g., the midplane or
Galactic center distance) or velocity (e.g., the 
     LSR) 
of the assumed coordinate system's origin. We use the epicyclic approximation, which
assumes that the motion in the $z$ and $R$ components of the orbit are
decoupled and follow simple harmonic motion about a circular and planar
     guiding orbit 
\citep[][Section~3.2 and references
therein]{2008gady.book.....B}. We refer to the radius of this orbit
     as the guiding radius $R_g$.
This approximation is an excellent description
of the thin disk and a good description of the thick disk in
an axisymmetric potential
    that ignores
the influence of the Galactic bar and
spiral arms. We also make the assumption of a perfectly flat circular
velocity curve with $v_c(R) = v_c$, a good approximation near the solar circle
\citep[e.g.,][]{2017MNRAS.465...76M}.

Under this approximation, we can write down the cylindrical components
of the orbits as
\beq\label{eq:orbits_epi}
\begin{split}
\phi(t) &= \Omega_c t \\
R(t) &= R_g + A_R \sin{(\kappa t + \alpha)} \\
z(t) &= A_z \sin{(\nu t + \beta)}
\text{,}
\end{split}
\eeq
where $\kappa$ and $\nu$ are the radial/epicyclic and vertical frequencies, $\Omega_c
\equiv v_c/R_g$ is the orbital frequency of the guiding center, $A_z$ and $A_R$ are the amplitudes in the vertical and radial coordinates, and $\alpha$ and
$\beta$ are 
    the initial orbital phases.
Similarly, the velocities of the orbit are given by:
\beq\label{eq:orbits_vel_epi}
\begin{split}
v_{\phi}(t) &= v_c \\
v_R(t) &= \kappa A_R \cos{(\kappa t + \alpha)} \\
v_z(t) &= \nu A_z \cos{(\nu t + \beta)}
\text{.}
\end{split}
\eeq

In this case, 
    the azimuthal action is
\citep[][Section~3.5.3b]{2008gady.book.....B}:
\beq\label{eq:Jphi_epi}
J_{\phi} = R_g v_c\text{,}
\eeq
and
    the radial and vertical actions are
\beq\label{eq:JR_Jz_epi}
\begin{split}
J_R &=  E_R/\kappa \\
J_z &=  E_z / \nu\text{,} 
\end{split}
\eeq
where $E_R$ and $E_z$ are the energy per unit mass in the radial and vertical
coordinates, respectively. Therefore,
\beq\label{eq:JR_Jz_epi_energy}
\begin{split}
J_R &= \frac{v_R^2 + \kappa^2 (R-R_g)^2}{2\kappa} \\
J_z &= \frac{v_z^2 + \nu^2 z^2}{2\nu}\text{.}
\end{split}
\eeq
Using Equations~\eqref{eq:orbits_epi} and \eqref{eq:orbits_vel_epi}, we can
simplify this:
\beq\label{eq:JR_Jz_epi_final}
\begin{split}
J_R &= \frac{\kappa A_R^2}{2} = \frac{v_{R,\text{max}}^2}{2\kappa} \\
J_z &= \frac{\nu A_z^2}{2} = \frac{v_{z,\text{max}}^2}{2\nu}\text{,}
\end{split}
\eeq
where the last equality in each line comes from the fact that
$v_{R,\text{max}} = \kappa A_R$ and $v_{z,\text{max}} = \nu A_z$.

Notice that while the value for each of $J_{\phi}$, $J_R$, and $J_z$ is phase
independent, the contribution from the kinetic and potential terms in
Equation~\eqref{eq:JR_Jz_epi_energy} is phase dependent. Now assume that the
coordinates $(R, z, v_{\phi}, v_R, v_z)$ are offset by $(\Delta R, \Delta z,
\Delta v_{\phi}, \Delta v_R, \Delta v_z)$. We can then apply the standard
propagation of errors formula to Equation~\eqref{eq:JR_Jz_epi_energy} to
determine the error in each of the actions. For $J_{\phi}$, the induced error
is:
\beq\label{eq:induced_Jphi}
\frac{\Delta J_{\phi}}{J_{\phi}} = \frac{\Delta R}{R_g}
                                    + \frac{\Delta v_{\phi}}{v_c}\text{.}
\eeq
For $J_R$, the induced error is:
\beq\label{eq:induced_JR}
\frac{\Delta J_R}{J_R} = \frac{2(R-R_g)}{A_R^2}\Delta R
                         + \frac{2v_R}{v_{R,\text{max}}^2} \Delta v_R \text{.}
\eeq
For $J_z$, the induced error is:
\beq\label{eq:induced_Jz}
\frac{\Delta J_z}{J_z} = \frac{2z}{A_z^2}\Delta z
                         + \frac{2v_z}{v_{z,\text{max}}^2} \Delta v_z \text{.}
\eeq
We have ignored second order contributions.

Since most of the time stars will be at maximum amplitude (i.e. turnaround) in
both $R$ and $z$, we can approximate the order of magnitude of the systematic
error in the actions by
\beq\label{eq:Ji_err_mosttime}
\begin{split}
\frac{\Delta J_{\phi}}{J_{\phi}} &= \frac{\Delta R}{R_g}
                                    + \frac{\Delta v_{\phi}}{v_c} \\
\frac{\Delta J_{R}}{J_{R}} &= \frac{2\Delta R}{A_R} \\
\frac{\Delta J_{z}}{J_{z}} &= \frac{2\Delta z}{A_z} \text{,}
\end{split}
\eeq
where we have again ignored second order terms.

In the remainder of this section, we compare our analytic estimates of the
effect of a midplane offset on actions against numerical calculations. A
numerical evaluation of the effect of velocity offsets on actions is deferred
to future work, as we discuss in Section~\ref{ssec:lsr_var}.

\subsection{Numerical Methods} \label{ssec:action_comp}
We now quantify the argument made in Section~\ref{ssec:cartoon} using
numerical computations of the actions for a range of orbits in a model
Galactic potential. We compute actions as in \citet{2018ApJ...867...31B},
using the code \texttt{gala} v0.3 to perform orbit integrations and conversion
to action space \citep{2017JOSS....2..388P,Price-Whelan:2018}. To compute
actions we use the torus-mapping technique first presented by
\citet{1990MNRAS.244..634M} and adapted by \citet{2014MNRAS.441.3284S} to
calculate actions for an orbital time-series starting from a phase-space
position $(x, v)$ and integrated in a potential $\Phi$. For our Galactic potential we use \texttt{MWPotential}, based on the Milky Way
potential available in \texttt{galpy} \citep{2015ApJS..216...29B}, which includes a Hernquist bulge and nucleus \citep{1990ApJ...356..359H},
a Miyamoto--Nagai disk \citep{1975PASJ...27..533M}, and 
    a Navarro, Frenk, \& White \citeyear{1997ApJ...490..493N} halo,
and is fit to empirically match some
observations.
We use the Dormand-Prince 8(5,3) integration scheme
\citep{Dormand80:integrator} with a timestep of $1\,\Myr$ and integrate for
$5\,\Gyr$, corresponding to $\sim 20$ orbits for a Sun-like star.

We assume the Sun is located at $(8.2, 0, 0)\,\kpc$. None of our orbit
integrations depend on the value of the LSR in this toy potential (though this
is important when using real data, since the conversion from heliocentric to
Galactocentric coordinates depends on the LSR). In this potential, we have
that the circular velocity $v_{\text{circ}}$ is $231\,\kms$ at the solar
circle.

Other methods for computing actions are used in the literature. For example,
the St\"ackel Fudge method \citep{2016MNRAS.457.2107S}, which uses a single
St\"ackel potential (with analytic actions) to approximate the Galactic
potential \citep{1985MNRAS.216..273D, 2012MNRAS.426.1324B}, was used in many
recent works exploring actions in the Galactic disk
\citep[e.g.,][]{2019MNRAS.484.3291T, 2018MNRAS.481.4093S, 2018arXiv180803278T}.
For disk-like orbits, existing implementations of the St\"ackel Fudge method
are of acceptable accuracy, but since we also consider halo-like orbits in
this work (where the St\"ackel Fudge method is inaccurate) we choose to use
orbit integration and torus mapping throughout \citep{2016MNRAS.457.2107S}.

\subsection{Quantification of the Midplane Effect} \label{ssec:quant}

\begin{figure*}[ht!]
\begin{center}
\includegraphics[width=\textwidth]{fig/schmactions_one_orbit.pdf}
\end{center}
\caption{The artificial phase-dependence in the computed actions with an
error in the Galactocentric coordinate system. We consider here \thick{},
which has actions of $(J_R, J_{\phi}, J_z) = (37.9, -1520,
7.0)\,\actunit$ and $z_{\text{max}}=850\,\pc$ (see Table~\ref{tab:orbits}). We
integrate the orbit according to the procedure laid out in
Section~\ref{ssec:action_comp}, and which we plot in
Appendix~\ref{app:orbits}. Then, we subtract $100\,\pc$ from the $z$ value
(upper panels) or the $x$ value (lower panels) of each position in the orbit,
corresponding to an erroneous observer assuming a midplane (upper) or solar
radius (lower) that is off by $100\,\pc$. We then allow an observer
to measure the orbit over $1\,\Gyr$ and perform the same orbit integration
procedure at each timestep, and report the values of the actions, with the true values given as horizontal dashed lines. The
computation of $J_{\phi}$ is pristine to errors in $z$, with only numerical
artifacts remaining. Only small errors are induced in $J_R$, with the middle
$90\%$ of values over the $\Gyr$ being within $\sim8\%$ of the true $J_R$. As
expected, large errors are induced in $J_z$ with a $100\,\pc$ offset in $z$,
with the middle $90\%$ of values being within $\sim43\%$ of the true $J_z$.
The $x$ offset induces uncertainties in $J_R$, $J_{\phi}$, and $J_z$ of
$\sim21\%$, $\sim3\%$, and $\sim3\%$.}
\label{fig:one_orbit_wrong_ref}
\end{figure*}

\begin{figure}
\begin{center}
\includegraphics[width=\columnwidth]{fig/schmactions_Jz_zerr_hist.pdf}
\end{center}
\caption{A histogram of the computed values of $J_z$ at different orbital phases for \thick{} (top
panel) and \thin{} (bottom panel) assuming a $z$ offset of
$100\,\pc$. One can see that if the observed $z$ values have a bias (from, e.g.,
an incorrectly computed midplane), then the induced error distribution in
$J_z$ is decidedly non-Gaussian. Therefore, any sort of error propagation must
take this into account. 
    A heuristic explanation
for the shape of each panel is given in the
text. We also plot one half the 95th percentile minus the 5th
percentile of each distribution as a horizontal arrow anchored on the true
$J_z$ value. We call this $\Delta J_z$ and will use it (along with the
similarly defined $\Delta J_R$ and $\Delta J_{\phi}$) to empirically describe
the error distribution. We see that $\Delta J_z$ roughly corresponds to the
distance from the true $J_z$ value to one of the modes of the distribution of
computed $J_z$ values. Similar plots for $J_R$ induced by a $z$ offset and
$J_R$ and $J_{\phi}$ induced by an $x$ offset are given in
Appendix~\ref{app:hist}.}
\label{fig:Jz_hist}
\end{figure}

We now quantify how a systematic error in the Galactocentric coordinate system
induces phase-dependence in the actions calculated from the observed position and velocity of a star. We consider
three orbits in the model potential described in
Section~\ref{ssec:action_comp} that are typical of stars in the thin
disk, thick disk, and halo. We summarize their initial
positions in phase space, the actions computed by integrating their orbits
in the correct potential, and other properties in Table~\ref{tab:orbits}. Each orbit, integrated
without systematic coordinate errors, is plotted in Appendix~\ref{app:orbits}.
We will refer to these orbits by their names (\thin{}, \thick{}, \halo{})
henceforth.

We begin with \thick{}. Consider an observer who can
measure the orbit's phase-space position at many different times
(and hence different orbital phases), but does so using a
coordinate system in which the midplane is systematically offset in height by
$100\,\pc$ from its actual location. To model this we subtract the vector $(0, 0,
100)\,\pc$ from each position in the orbit. This corresponds to an
observer physically located at, e.g., the position $(8, 0, 0)\,\kpc$ in the
coordinate system of the true potential, but erroneously thinking they are
located at $(8, 0, 0.1)\,\kpc$.

We consider the observer making a measurement, integrating an orbit, and
computing actions every 
    megayear 
using the prescription above. However, we
specify the star's starting position using the systematically offset
coordinate system.
Essentially we are shifting and then reintegrating at each point along the original orbit.
The actions computed using the offset coordinate system
for each phase-space starting point are shown for the first 
  gigayear 
of the orbit in the upper panels of
Figure~\ref{fig:one_orbit_wrong_ref}.\footnote{Occasionally the numerical
scheme fails and very large actions are reported by \texttt{gala}---we perform
a $5\sigma$ clip on each action to exclude such orbits, but this only excludes
a total of $5$ orbits out of the $1000$ considered for
Figure~\ref{fig:one_orbit_wrong_ref}. Some numerical artifacts remain, but the
vast majority of orbits are computed properly.}

We also perform the same procedure in the lower panels but assuming an $x$
component offset of $100\,\pc$, i.e., subtracting the vector $(100, 0,
0)\,\pc$. This is equivalent to a measurement error in the distance from the
Sun to the Galactic center.

Figure~\ref{fig:one_orbit_wrong_ref} shows that the actions computed in the
offset coordinate systems oscillate
as a function of the time/orbital phase at which the
star's phase-space position is observed.
This time dependence comes even though
the observer is using the correctly constructed, best-fit, static, axisymmetric
potential. The relative size of the phase variation in each action depends on
the direction of the systematic offset as well as the true values of the
actions (i.e. the type of orbit). In reality, we will have one measurement of
the phase-space position to work with, in which case the determination of the
orbital phase in $R$ or $z$ is degenerate with the degree of systematic offset
in that coordinate (see Figure~\ref{fig:cartoon}). In the following we
therefore quote percentile ranges for the possible values computed for each
action as a proxy for the effect of these systematic errors in the coordinate
system.

For a systematic offset in $z$ (upper panels), the 95\textsuperscript{th}
minus 5\textsuperscript{th} percentiles are $2.2$ and $6.2\,\actunit$ for
$J_R$ and $J_z$, respectively. These are fractional errors of $5.7\%$ and
$86\%$, respectively. The error induced in $J_{\phi}$ is negligible, as
expected since $J_{\phi}$ only depends on the $x$- and $y$-components of the
position and velocity of the stars.\footnote{In practice, however, $J_{\phi}$
is computed as part of the torus-fitting method.} It is worth pointing out
that a $100\,\pc$ error in an orbit with $z_{\text{max}}=850\,\pc$\,---\,a
$12\%$ error\,---\,induced an $86\%$ error in the computation of $J_z$.

For a systematic offset in $x$ (or distance to the Galactic center), the
95\textsuperscript{th} minus 5\textsuperscript{th} percentiles are $6.9$,
$47$, and $0.71\,\actunit$ for $J_R$, $J_{\phi}$ and $J_z$, respectively.
These are fractional errors of $21\%$, $3.1\%$ and $3.1\%$ in these actions, respectively, despite only a $1.2\%$ error in the distance to the Galactic center.

In Figure~\ref{fig:Jz_hist}, we plot a histogram of the values of
$J_z$ computed at different orbital phases for \thick{} (top panel) and \thin{} (bottom panel),
assuming a $z$ offset of $100\,\pc$ (as in the upper right panel of
Figure~\ref{fig:one_orbit_wrong_ref}). The true value is plotted as a
vertical dashed line. The systematic error in $J_z$ induced by a
systematic offset in $z$ is non-Gaussian and bimodal; neither of the modes is centered on the null value. In the case of \thin{}
(bottom panel), we see that, in addition to the prior complications, the
distribution is not even centered on the true value. This comes about when the midplane
error is~$\gtrsim z_{\text{max}}$, where $z_{\text{max}}$ is the maximum
height of the orbit (equivalent to $A_z$ in the epicyclic approximation, see
Section~\ref{ssec:epi_action}).

In Appendix~\ref{app:hist} we plot the same histogram as in
Figure~\ref{fig:Jz_hist}, but for the distributions of $J_R$ induced
by a $z$ offset (upper left panel of Figure~\ref{fig:many_orbit_wrong_ref})
and the distributions of $J_R$ and $J_{\phi}$ errors induced by an
$x$ offset (lower left and lower center panels of
Figure~\ref{fig:many_orbit_wrong_ref}, respectively). We find similar error
distributions as in Figure~\ref{fig:Jz_hist}, with the exception that the
computed $J_R$ distribution induced by an $x$ offset more closely resembles a
Gaussian distribution.

\begin{figure*}[htb!]
\begin{center}
\includegraphics[width=\textwidth]{fig/schmactions_many_orbits.pdf}
\end{center}
\caption{We report one half the 95th minus 5th percentile of the error
in the measured action ($\Delta J_i$) from coordinate system errors for
\thin{}, \thick{}, and \halo{} orbits (Table~\ref{tab:orbits}). See discussion in the
text and Figure~\ref{fig:Jz_hist} for the justification in using this to measure the magnitude of the
induced error. The left, center, and right panels show the result for $J_R$,
$J_{\phi}$, and $J_z$, respectively. The upper panels consider an offset in
$z$ and the lower panels consider an offset in $x$ (equivalently, an offset in
the solar radius). In some panels we also plot as dashed lines the epicyclic
prediction of the induced action error (Equation~\eqref{eq:Ji_err_mosttime}).
In the epicyclic approximation, a $z$ offset only induces an error in $J_z$
--- for all three orbits the epicyclic approximation is a good description of
the $J_z$ error. An $x$ offset induces an error in $J_R$ and $J_{\phi}$. The
error in $J_R$ is somewhat well-described for \thin{} and \thick{},
and a poor description for \halo{}. For $J_{\phi}$, the
epicyclic approximation is not a good description for any orbit.}
\label{fig:many_orbit_wrong_ref}
\end{figure*}

We now 
    suggest a heuristic explanation for
the shape of Figure~\ref{fig:Jz_hist}.
Consider first the \thick{} (top panel), where the offset in $z$ is
much less than $z_{\text{max}}$ of the orbit. The peaks in the distribution
correspond to the turning points of the orbit (or points of maximum vertical
amplitude), where $v_z \sim 0$ and where the star is on most of it orbit. This
is why the distribution, which is calculated at evenly spaced time intervals, peaks at these values. For \thin{}
(bottom panel), the offset in $z$ is comparable to $z_{\text{max}}$. Now,
there will be some points in the orbit where $v_z = 0$ and $z=0$ (in the
erroneous coordinate system). At these points, the computed $J_z$
will vanish. The asymmetry and systematic offset then comes about because of
the constraint that $J_z \geq 0$.\footnote{This argument is similar to ones given in
cosmology for why gravity produces non-Gaussianity in the density field, since
the density cannot become negative but it can grow arbitrarily large.}

Gaussian summary statistics are clearly insufficient to describe the distribution shown in
Figure~\ref{fig:Jz_hist}. We therefore elect to measure this error by computing one
half the 95th percentile minus the 5th percentile of the distribution
of action values. We refer to this quantity as $\Delta J_i$ for each action and plot it in Figure~\ref{fig:Jz_hist} as a horizontal arrow anchored on the true action value.
 Because of the bimodality
of the error distribution, this quantity roughly measures the distance from
the true action value to the peak of one of the modes. Furthermore, this
bimodality also implies that $\Delta J_i$ is not very sensitive to the exact
percentiles used. This summary statistic does not reflect the bias induced
when the midplane error is~$\gtrsim z_{\text{max}}$.

We now repeat the same procedure as in Figure~\ref{fig:one_orbit_wrong_ref}
but for systematic offsets between $0$ and $500\,\pc$ in the $z$ and $x$
components. In Figure~\ref{fig:many_orbit_wrong_ref}, we report $\Delta
J_i/J_i$ for the three different fiducial orbits in Table~\ref{tab:orbits}.
Figure~\ref{fig:one_orbit_wrong_ref} considered \thick{} (\thickcolor), but we also now consider the effect on
the action determined for \thin{} (\thincolor) and \halo{} (\halocolor).

The upper panels of Figure~\ref{fig:many_orbit_wrong_ref} shows the spread
induced in each action for an offset in the $z$-component. In the lower panels we consider offsets in the $x$ component (i.e. the
solar radius). The left, center, and right columns show the fractional spread
in the values computed for $J_z$, $J_{\phi}$, and $J_R$, respectively.

In the upper middle panel of Figure~\ref{fig:many_orbit_wrong_ref}, there is
essentially no spread in the determination of $J_{\phi}$. This is expected
since $J_{\phi}$ is independent of $z$ and is thus unaffected by offsets in
$z$, as discussed earlier. Indeed, the result we found in
Figure~\ref{fig:one_orbit_wrong_ref} for \thick{} holds for all
orbit types. This is also a demonstration of the robustness of the
integration and action calculation methods we use.

The upper right panel of Figure~\ref{fig:many_orbit_wrong_ref} shows that the
fractional error in $J_z$ is more exaggerated for more planar (disk-like)
orbits. For \thin{}, 
a systematic offset of $15\,\pc$
in the $z$-coordinate results in a $25\%$ deviation in $J_z$,
     while a $120\,\pc$ offset results in the same deviation for \thick{}.
We find that \halo{} is relatively resistant to errors in the midplane, with
only $\sim15\%$ error in $J_z$ out to an offset of $500\,\pc$.

For the offset in the solar radius (lower panels), the error is largest for
$J_R$, with some deviations resulting in $J_{\phi}$ and relatively small
deviations in $J_z$. In the lower middle and lower right panels all three
lines nearly overlap.

In each panel of Figure~\ref{fig:many_orbit_wrong_ref}, where relevant, we
include the estimation of the action errors derived under the epicyclic
approximation from Equation~\eqref{eq:Ji_err_mosttime}, with $\Delta
v_{\phi}=0$, as dashed lines in the color of each orbit. This equation is
relevant since during most of the orbit the star will be 
    close to 
maximum radial and
vertical amplitude. Note that we
consider an error in the $x$-coordinate $\Delta x$, which is not exactly the
same as $\Delta R$. For observations of stars close to us, we have that
$\Delta x \sim \Delta R$, but for the experiment performed in this section we
consider observations of the star throughout its entire orbit. This introduces
a factor of $2/\pi$ when converting from $\Delta x$ to $\Delta R$, which we
derive in Appendix~\ref{app:deltax}.

The epicyclic approximation is a good predictor of $\Delta J_z$, even for \halo{}.
It performs similarly for $\Delta J_R$, now underpredicting for \halo{}
and slightly overpredicting for \thin{}. Note that for the
particular orbits we chose, \thin{} has slightly larger $A_R$ than
\thick{}, and so we actually expect the epicyclic approximation to perform
slightly worse for \thin{} in this case. The epicyclic approximation underpredicts $\Delta J_{\phi}$
for all orbits.

\begin{figure}
\begin{center}
\includegraphics[width=\columnwidth]{fig/schmactions_many_orbits_Jz_fun.pdf}
\end{center}
\caption{The fractional error in $J_z$ as a function of $J_z$ for a few
different offsets in $z$. All orbits have the same initial position of $(8, 0,
0)\,\kpc$ and velocity $(0, -190, v_z)\,\kms$, where we vary $v_z$.\footref{note:vz_orbits} We show
this for a $z$ offset of $10$, $50$, and $100\,\pc$ (orange, teal, and green,
respectively). As before, the error ($\Delta J_z$) is one half the 95th
minus 5th percentile of the distribution of $J_z$ values over the
course of the orbit. There are large errors for \thin{}-like orbits ($J_z
\sim 0.5\,\actunit$), even for a small midplane offset of $10\,\pc$. As dashed
lines in each color we also plot the prediction for $\Delta J_z/J_z$ from the
epicyclic approximation (Equation~\eqref{eq:Ji_err_mosttime}), which shows
excellent agreement with the numerically computed values.}
\label{fig:dJz_fun_Jz}
\end{figure}

To further understand the effect of the midplane error, we also plot the
fractional error in $J_z$ as a function of $J_z$ for $z$ offsets of 
     $10$, $50$, and $
100\,\pc$ (orange, teal, and green, respectively) in
Figure~\ref{fig:dJz_fun_Jz}. 
For each orbit, we set the initial position to be
$(8,0,0)\,\kpc$ and the initial velocity to be $(0, -190, v_z)\,\kms$, where
we vary $v_z$.\footnote{\label{note:vz_orbits}One can recover \thin{},
  \thick{}, and \halo{} by setting $v_z=10$, $50$, and $190$,
 respectively, 
      giving $J_z \simeq 0.7, 20,$ and $500\,\actunit$  
(see Table~\ref{tab:orbits}).} For a \thin{}-like orbit ($J_z\sim0.7\,\actunit$), 
even a
$10\,\pc$ offset in $z$ is enough to induce a 
         $\sim20\%$ 
error in $J_z$. For
larger values of $J_z$, the fractional errors are suppressed, but the induced
error can still be large depending on how great the $z$ offset is. We also
plot the epicylic prediction for $\Delta J_z / J_z$ from
Equation~\eqref{eq:Ji_err_mosttime} as dashed lines for each $z$ offset. We
find that the epicyclic approximation matches the numerical estimate quite
well.

\section{
Azimuthal Midplane Variations}
\label{sec:local_fire}

The stellar midplane of the Galaxy should vary as a function of azimuth and
Galactocentric radius due to small, local variations in the stellar density.
Hints of this variation as a function of Galactocentric radius have been noted
through their impact on the stellar velocity distribution pre-\textit{Gaia} by
\citet{2012ApJ...750L..41W}, \citet{2013ApJ...777L...5C}, and
\citet{2013MNRAS.436..101W} and recently post-\textit{Gaia} by
\citet{2019arXiv190209569F}. As pointed out by, e.g.,
\citet{2014ApJ...797...53G} and \citet{2019ApJ...871..145A}, among many
others, the gas distribution in the Galaxy also shows significant density
variation across the disk.

The local Galactocentric coordinate system is defined based on the location of
the Sun relative to the midplane. Extending this coordinate system to a global
one therefore introduces systematic errors in the $z$ components of stellar
positions. As discussed in Section~\ref{sec:ref_frame}, this systematic error
introduces errors in integrating orbits and computing actions.

We specifically consider \emph{azimuthal} variations in the midplane at the solar circle, as defined by the stellar mass density. Since, to our knowledge, there are no direct empirical measurements of these variations in the Milky Way, we use example simulations from two classes of simulations to estimate the size of this effect.

One set are three zoom-in, cosmological hydrodynamical simulations of isolated Milky Way-mass galaxies from the FIRE collaboration, described briefly in Section~\ref{ssec:cosmozoom}. These include stars, gas, and dark matter in a fully cosmological setting but are not tailored to specific properties of the Milky Way (such as the scale height or scale length, or the details of the accretion history). We use these simulations to span the range of possibilities for azimuthal midplane variations.

The other set of simulations are isolated N-body simulations of interactions between the Milky Way and a Sagittarius-like dwarf galaxy, described briefly in Section~\ref{ssec:sag_sim}. These include dark matter and stars and are tailored to existing measurements of the structure of the Milky Way's disk and of the orbit and properties of the Sagittarius dwarf galaxy. Comparing the azimuthal midplane variations in the host galaxy of these simulations before and after the interaction with the Sagittarius-like object gives an idea of the effect of one minor merger whose properties are relatively well measured. Azimuthal variations of the mean vertical height of stars has been explicitly pointed out in a different simulation of a Sagittarius-like encounter by \citet{2013MNRAS.429..159G}.

\subsection{Description of FIRE Simulations} \label{ssec:cosmozoom}
The FIRE cosmological hydrodynamic simulations
\citep{2014MNRAS.445..581H,2018MNRAS.480..800H} use the zoom-in technique
\citep[e.g.,][]{1993ApJ...412..455K,2014MNRAS.437.1894O} to model the formation
of a small group of galaxies at high resolution in a full cosmological
context. Feedback from supernovae, stellar winds, and radiation from massive
stars is implemented at the scale of star forming regions following stellar
population synthesis models, generating galactic winds self-consistently
\citep{2015MNRAS.454.2691M, 2017MNRAS.470.4698A} while reproducing many
observed galaxy properties, including stellar masses, star formation
histories, metallicities, and morphologies and kinematics of thin and thick
disks \citep{2014MNRAS.445..581H, 2016MNRAS.456.2140M, 2017MNRAS.467.2430M,
2016ApJ...827L..23W, 2018MNRAS.481.4133G, 2018MNRAS.480..800H}.

For this work,
we focus on the three Milky Way-mass zoom-ins considered in
\citet{2018arXiv180610564S}, which were simulated as part of the
\textit{Latte} suite and show broad agreement of many of their global
properties with observations of the Milky Way \citep{2016ApJ...827L..23W,
2018MNRAS.481.4133G}. The 
$\z = 0$ snapshots\footnote{In this work, to avoid
confusion with the vertical height $z$, we refer to cosmological redshift as
$\z$.} of these three simulations, 
      The snapshots of these three simulations at cosmological redshift $\z = 0$,
named \mi{}, \mf{}, and \mm{},
are publicly available alongside associated mock \textit{Gaia} DR2
catalogues generated from them.\footnote{\url{http://ananke.hub.yt}}

\begin{deluxetable*}{cCCCCC}[htb!]
\tablecaption{Stellar and gas disk scale heights of the Milky Way and the
FIRE galaxies considered in this work (described in Section~\ref{ssec:cosmozoom}). For comparison, we also give the median
softening lengths for the FIRE galaxies, computed for cold gas ($T<1000\,\text{K}$) and stars with $\abs{R-R_0} < 0.5\,\kpc$ and $\abs{z} < 1\,\kpc$. We have assumed that $R_0=8.2\,\kpc$.\label{tab:scale_height}}
\tablehead{\colhead{galaxy} & \colhead{\makecell{cold\tablenotemark{a} gas disk \\ scale height}} &
\colhead{\makecell{thin disk \\ scale height}} & \colhead{\makecell{thick disk \\ scale height}} & \colhead{\makecell{cold\tablenotemark{a} gas \\
softening length}} & \colhead{\makecell{stellar \\ softening length}} \\
\colhead{} & \colhead{(pc)} & \colhead{(pc)} & \colhead{(pc)} & \colhead{(pc)}
& \colhead{(pc)} }
\startdata
Milky Way\tablenotemark{b} & 40 & 300 & 900 & \nodata & \nodata \\
\mi\tablenotemark{c} & 800\tablenotemark{d} & 480 & 2000 & 53.4 & 11.2 \\
\mf\tablenotemark{c} & 360 & 440 & 1280 & 57.2 & 11.2 \\
\mm\tablenotemark{c} & 250 & 290 & 1030 & 60.1 & 11.2 \\
\enddata
\tablenotetext{a}{$T<100\,\text{K}$}
\tablenotetext{b}{\citet{2008ApJ...673..864J,2016ARAA..54..529B}}
\tablenotetext{c}{\citet{2018arXiv180610564S}}
\tablenotetext{d}{The azimuthally averaged gas vertical density profile in
\mi{} is nearly constant to this height, though individual regions show smaller
scale heights and dense clouds.}
\end{deluxetable*}

These simulations contain dark matter particles of mass $\sim35,000\,\Msun$,
gas particles of mass $\sim7000$ to $20,000\,\Msun$, and star particles of
mass $\sim5000$ to $7000\,\Msun$, with the lower end coming from
stellar evolution \citep{2018arXiv180610564S}. Softening lengths for dark
matter and star particles are fixed at $112\,\pc$ and $11.2\,\pc$,
respectively.\footnote{This is $2.8$ times the often-quoted
Plummer-equivalent.} The gas softening length is adaptive, but at $\z=0$ the
median softening length for cold ($T < 100\,\text{K}$) gas particles around
roughly solar positions (with galactocentric cylindrical radii within
$0.5\,\kpc$ of $8.2\,\kpc$ and $\abs{z}<1\,\kpc$) is $53.4$, $57.2$, and
$60.1\,\pc$ for \mi{}, \mf{}, and \mm{}, respectively. These values are
summarized in Table~\ref{tab:scale_height}, along with measurements of the
stellar and gas disk scale heights.

The softening lengths used in the simulations can affect the ability to
resolve the very thinnest planar structures, which in turn can affect how much
the density-based midplane varies as a function of azimuth. The Milky Way's
dense, star-forming gas disk is thought to have a scale height of about
$40\,\pc$, on the order of the cold gas softening length
\citep{2019ApJ...871..145A}. The thin stellar disk has a scale height of about
$300\,\pc$, $\sim30$ times the stellar softening length
\citep{2008ApJ...673..864J}. We therefore expect that resolution effects are
still affecting the scale heights of these components in the simulations,
especially the cold gas. Indeed, the stellar scale heights of the simulated
galaxies are equal to or larger than the Milky Way's while the gas scale
heights are significantly larger (although the proper basis comparison is less
clear in the case of the gas; the quoted value for the Milky Way comes from
studies of high-mass star-forming regions). 
     The 
midplanes defined by gas and stars can be tilted with respect to one another as well,
precluding extending the precision of the gas midplane definition to the
stellar component.

Cosmological simulations of Milky Way-mass galaxies are not perfect
representations of the true Milky Way in other ways as well,
       as discussed in \citet{2018arXiv180610564S}.
The failure of cosmological simulations to exactly reproduce the Milky
Way is not necessarily due to limitations of the numerical
model. Candidate Milky Way-like galaxies are chosen solely on their
mass and isolation, for which there are a wide variety of possible galaxies with qualitatively different
properties. 
       For example,
the velocity structure of \mi{} is closer to M31's than the Milky
Way's (S. Loebman et al., in preparation). 

However, in this work we are most interested in the global properties of the
potential, and specifically in deviations from axisymmetry. From this
perspective, the simulated galaxies are actually \emph{more} axisymmetric than
we might expect of the Milky Way. While they have prominent spiral arms, none
has as strong a bar as the Milky Way does at present day, and none has a
nearby companion like the Large Magellanic Cloud. One of the three we consider
(\mf) does have an ongoing interaction with a satellite galaxy similar in mass to
Sagittarius, which has punched through the Galactic disk outside the solar
circle, leaving behind some of its stars and inducing warping in the disk.

In this work, we take the galactocentric coordinate system described in
Section 3 of \citet{2018arXiv180610564S} as our fiducial coordinate system for
each galaxy. In short, the center of the galaxy is found iteratively. The
center of mass velocity is then determined by all star particles within
$15\,\kpc$ of this center. The galaxy is then rotated onto a principal axis
frame determined by stars younger than $1\,\Gyr$ inside of the fiducial solar
radius $R_{0} = 8.2\,\kpc$, such that the disk plane is the $x$--$y$ plane.

\subsection{Description of Milky Way-Sagittarius Interaction Simulation}
\label{ssec:sag_sim}
In addition to the cosmological zoom-ins, we will also briefly consider
results from a live N-body simulation of a Sagittarius-like encounter. This
simulation offers us the ability to see how the midplane varies in a more
controlled environment. The 
     simulation is described by
\citet{2018MNRAS.481..286L}, but we briefly summarize the most relevant
details here.

For the Milky Way, the dark halo is modeled as a Hernquist
sphere of mass $10^{12}\,\Msun$ and scale length of $52\,\kpc$, the
disk
    is
modeled as an exponential disk with a scale radius of $3.5\,\kpc$, scale
height $0.53\,\kpc$, and mass $6\times10^{10}\,\Msun$, and a Hernquist bulge
of mass $10^{10}\,\Msun$ and scale length $0.7\,\kpc$
\citep{1990ApJ...356..359H}
     is included.
     The Sagittarius dwarf is
modeled with two components: a dark matter Hernquist sphere of mass $8\times10^{10}\,\Msun$ and
scale length $8\,\kpc$, and a stellar component modeled as a Hernquist
sphere of mass $6.4\times10^8\,\Msun$ and scale length $0.85\,\kpc$. All
components are realized with 
    distributions of live
N-body particles; the Milky Way and Sagittarius are each initialized
to be in equilibrium in isolation. 
 
The mass resolution of the simulation is 
       $2.6\times10^4$, $1.2\times10^4$, 
and $1.0\times10^4\,\Msun$ for the dark matter, disk, and bulge components,
respectively. For the disk and bulge components, a softening length of
$30\,\pc$ is used whereas for the halo a softening length of $60\,\pc$ is
used. For Sagittarius, the softening length for the dark matter and
the stars is 
     $60$ 
and $40\,\pc$, respectively.

The fiducial coordinate system for these N-body simulations is the rest frame
of the aligned host galaxy at the beginning of the simulation.

\subsection{The Local Midplane} \label{ssec:local_midplane}
Using the two sets of simulations, we determine the local midplane
       as a function of azimuth at the solar circle
that an observer might measure if they were situated in each of these
galaxies% 
. Starting from the coordinate system
described in the previous section, which is 
aligned so that the
$z$-coordinate is approximately perpendicular to the disk plane at the solar circle, we place our imaginary
observer at $z=0$ and a galactocentric cylindrical radius of $8.2\,\kpc$ and
vary the azimuth between $0<\phi<2\pi$. 
At 
    equally spaced values of 
$\phi$ we then compute the median $z$ for stars within a cylinder of radius $0.5\,\kpc$ and
height $1\,\kpc$ perpendicular to the fiducial disk  
     and centered on it. We choose to use 50 values, sufficiently few that no cylinder shares stars
     with its neighbors.
We then re-define the new
midplane of the cylinder to be the median $z$, re-select stars, and iterate
until the median $z$ value converges. We find that only $10$ iterations of
this procedure are necessary for convergence. The resulting median $z$ is
taken to be what our observer would measure as the local Galactic midplane at
each $\phi$.

\begin{figure*}[htb!]
\begin{center}
\includegraphics[width=\textwidth]{fig/midplane_fit.pdf}
\end{center}
\caption{The local midplane determined at the fiducial solar circle
($R_0 = 8.2\,\kpc$) for the three FIRE galaxies \mi{}, \mf{}, and \mm{} (left,
center, and right panels) as a function of azimuthal angle, at cosmological redshift $\z =0$. The local midplane is determined at
a position $\phi$ by taking the median height of all stars within
$R=0.5\,\kpc$ and $z=1\,\kpc$ (in cylindrical coordinates). 
In order to allow for the possibility that the fiducial
galactocentric coordinate system is incorrect, we subtract the best fit
     sine
curve from each panel. We then bootstrap resample $1000$ times to determine
$1\,\sigma $ error bars, which we report as dashed lines.}
\label{fig:midplane}
\end{figure*}

\begin{figure*}[htb!]
\begin{center}
\includegraphics[width=342.078286667pt]{fig/midplane_fit_chervinsim.pdf}
\end{center}
\caption{The local midplane determined at the fiducial solar circle
($R_0 = 8.2\,\kpc$) for four different time steps from live N-body simulations of a
Sagittarius encounter with the Milky Way \citep{2018MNRAS.481..286L}. As
before, we have subtracted the best fit 
     sine
curve to account
for inaccuracies in the galactocentric coordinate system. Error bars are
calculated as in Figure~\ref{fig:midplane}. The upper panels
show the midplane as a function of azimuth before the first encounter near the
solar circle at $t=2.0\,\Gyr$ and $t=4.0\,\Gyr$, with an encounter happening close to the solar circle near $t=6.0\,\Gyr$. The fact that the $t=6.9\,\Gyr$ panel, which shows the midplane variation after some relaxation,
looks qualitatively similar to the panels from the FIRE simulations
(Figure~\ref{fig:midplane}) is evidence that midplane variations are
generated, in part, by mergers.}
\label{fig:midplane_chervin}
\end{figure*}

This procedure assumes perfect density estimation, and therefore perfect
corrections for extinction within the cylinder defining the ``solar
neighborhood.'' Imperfect extinction correction is likely to increase the
amplitude of the estimated fluctuations in $z$.

To account for the effect of particle noise, we bootstrap resample stars
within a cylinder of height $2\,\kpc$ and the same radius $1000$
times and determine the $1\,\sigma$ error bars by repeating the midplane
determination with that reselection.

To allow for potential small inaccuracies in the determination of the original
fiducial coordinate system, we also subtract the best fit 
    curve of the form 
    \beq A \sin{\left(\phi + B\right)} + C \label{eq:fit-curve}\eeq 
from the midplane as a
function of azimuth to account for an overall tilt of the midplane (a
simplified version of the strategy described in
\citealt{2019ApJ...871..145A}). 
%mm [perhaps this would be better presented as a table] For
%$\text{A}$, the values are $-165$, $45$, and $8.8\,\pc$, for
%$\text{B}$ the values are $0.67$, $-0.088$, and $0.031\,\text{rad}$
%and for $\text{C}$ the values are $-69$, $19$, and $-18\,\pc$ for
%\mi{}, \mf{}, and \mm{}, respectively. 
    For models (\mi{}, \mf{}, \mm{}) the best fit values are $A =
    (-165, 45, 8.8)\,\pc$, $B = (38, -5.0, 1.8)\,\text{deg}$,
    and $C = (-69, 19, -18)\,\pc$.
For the assumed solar radius of $8.2\,\kpc$, we can approximate the
angle offset 
    $\Delta \theta$ 
for the $z$-axis from the values of 
      $A$. We find for the same models $\Delta \theta = (1.15, 0.31,
      0.062)\,\text{deg}$.
These angle offsets are consistent with the
values given in \citet{2018arXiv180610564S} for the difference between the
$z$-axis as defined by the gas and stars.

Figure~\ref{fig:midplane} shows the relative $z$ location of the inferred
midplane an observer would determine as a function of azimuth for
each galaxy, using their local solar neighborhood (the cylinder defined
above). The
$1\,\sigma$ error from the bootstrap procedure is shown as the dashed-line
error bars. The middle $90\%$ of midplane values across the solar circle spans
    $(1.9 \times 10^2, 1.6 \times 10^2, 84)\,\pc$ for these models.
In two of the three cases
the midplane therefore varies by more than $\pm 100\,\pc$ depending on the
azimuth along the solar circle; in the third (\mm{}, which has the thinnest
``thin disk'' of stars, but the largest stellar mass) the variation is closer
to~$\pm 50\,\pc$.

We compute the same midplane variation in Figure~\ref{fig:midplane_chervin},
but for four succcessive timesteps of the live N-body simulation of a
Sagittarius encounter \citep{2018MNRAS.481..286L}. Again we have subtracted a
best fit curve of the form 
    given in Equation~(\ref{eq:fit-curve}), with the values at times
    $(2.0, 4.0, 6.0, 6.9)\,\Gyr$ being $A = (9.5, 2.5, -2.1 \times 10^2,
    -3.9 \times 10^2)\,\pc$, $B
    = (0.074, 0.039, -36,
    -57)\,\text{deg}$, and $C = (2.6,-7.8, -65, -53)\,\pc$.
The middle $90\%$ of midplane values across the solar circle spans 
       $(49, 62, 140, 120)\,\pc$.

These values for the midplane variation are consistent with the azimuthal
midplane variations seen by \citet{2013MNRAS.429..159G}. However, they only
saw significant variations in their Heavy but not their Light Sagittarius model (virial masses of $10^{11}\,\Msun$ and
$\sim3\times10^{10}\,\Msun$, respectively). The model we used (L2 from
\citet{2018MNRAS.481..286L}) has a virial mass of $6\times10^{10}\,\Msun$, intermediate between their two models.

In the upper panels, we see that the midplane is relatively flat in the inner
galaxy, but additional encounters drive strong midplane variation. In the
lower left panel, we see a strong $m=2$ mode develop, consistent with the
$R=8\,\kpc$ panel of Figure~17 in \citet{2018MNRAS.481..286L} ($m=0$ and $m=1$ modes are stronger, but these
are removed in our sine-curve subtraction). The lower right
panel, which shows the galaxy at $t=6.9\,\Gyr$ when some relaxation has occured, is
qualitatively similar to the midplane variations we saw in the FIRE
simulations (Figure~\ref{fig:midplane}), evidence that they are at least
partially driven by mergers.

\subsection{Velocity Variations} \label{ssec:lsr_var}
We also expect that the LSR should vary as a function of azimuth. We perform
this calculation in Appendix~\ref{app:lsr} to estimate the components of the
LSR as a function of azimuth, but performing a best-fit subtraction to correct
for misalignment of the original coordinate system (as in the previous
section) is more involved. Since we find that the variation in the LSR is less
pronounced than for the midplane, and since offsets in velocity only
contribute to second order to $\Delta J_R$ and $\Delta J_z$ when a star is at
maximum amplitude in $R$ or $z$ (where the majority of the orbit is, see
Section~\ref{ssec:epi_action}), we defer this calculation to future work.

\section{Discussion} \label{sec:discussion}
We have used high-resolution simulations to illustrate 
    why
we expect the local midplane defined by stellar density to vary with azimuth by up to $\pm
100$ pc, as a natural consequence of the non-axisymmetry of the Galactic disk
at small scales. While this is not in itself surprising or new, we have also
    argued
that the discrepancy between our local midplane and that of distant
stars introduces a systematic error in the $z$ component when converting from
heliocentric to Galactocentric coordinates. This systematic error introduces a
non-Gaussian error in the vertical action, $J_z$, when starting from the
present-day positions and velocities of stars as measured by, e.g.,
\textit{Gaia}.

These systematic errors are most important for stars on thin disk-like orbits,
where they can be large enough to yield actions representative of orbits in
the thick disk. This effect is entirely due to the extension of a local to a
global coordinate system, and is separate from real diffusion in stellar
integrals of motion caused by interactions with these same deviations from
axisymmetry, such as resonant perturbation by spiral arms or scattering from
molecular clouds \citep{2014RvMP...86....1S}.

\subsection{Estimates of Milky Way Midplane Offsets 
      }

\label{ssec:mw_data_midplane}

Systematic variations in $v_z$ and number were first noted as asymmetries in the local
velocity distribution towards the North and South Galactic Caps from the
radial velocity surveys of the Sloan Digital Sky Surveys
\citep{2012ApJ...750L..41W} and RAdial Velocity Experiment
\citep{2013MNRAS.436..101W}. Subsequently, \citet{2013ApJ...777L...5C} pointed
out suggestions of an oscillation in average vertical velocities of order
$5\,\kms$ on 
    roughly kiloparsec 
scales looking toward the Galactic anticenter.

Work
by the \textit{Gaia} collaboration confirmed these preliminary results on the
velocity and spatial scales of 
oscillation with clear spatial maps made
using DR2 data of median $v_z$ over a significant Galactic volume
\citep{2018A&A...616A..11G, 2019arXiv190209569F}, which can be explained with models of Sagittarius-like encounters \citep{2013MNRAS.429..159G,2018MNRAS.481..286L,2019MNRAS.485.3134L}. We also see in the FIRE
simulation that the vertical velocity variation as a function of azimuth is
$\sim5\text{--}10\,\kms$ (Figure~\ref{fig:lsr_variations}), consistent with
these observations.

The vertical frequency of \thin{} and \thick{} are $\sim
0.09\,\Myr^{-1}$ and $\sim0.06\,\Myr^{-1}$, respectively (Table~\ref{tab:orbits}). By dimensional
analysis, and assuming a vertical velocity variation of $5\textup{--}10\,\kms$, we
therefore expect the midplane offsets to be $\sim 57\textup{--}170\,\pc$. We
stress that this is a rough calculation.

Three-dimensional dust maps also offer a view into the expected variation of
the stellar disk, since dust should trace 
    regions of massive star formation.
Figure~9 of \citet{2019MNRAS.483.4277C}, Figure~1 of
\citet{2019arXiv190105971L}, and Figure~2 of \citet{2019arXiv190502734G} all
show that the midplane varies by $\sim10\degree$ at a distance of
$\sim0.75\,\kpc$, corresponding to a physical 
        vertical variation 
of $\sim130\,\pc$.

Already we see evidence in the data from velocities and dust maps for midplane
offsets on the order of what we saw in both sets of simulations.

\subsection{Uncertainties in the Solar Position and Velocity}\label{ssec:coord_off}

Uncertainties in measurements of the position and velocity of the Sun relative
to the Galactic center can also contribute to systematic error in the actions, since converting from heliocentric to Galactocentric
coordinates relies on these measurements. Therefore, errors in their values
will induce a systematic offset in the Galactocentric phase-space position of
any observed star. Considerable effort has been placed on each of these measurements, but uncertainties remain, and detailed modeling across the disk\,---\,particularly for dynamically cold stars\,---\,may have to take them into
account.
Here we briefly review the current measurements
   of the four relevant quantities, 
their uncertainties, and the implications for the calculation of actions.

\subsubsection{Galactic Center Position}
First, one must define the center of the Galaxy. This is usually taken to be
the location of the central supermassive black hole, Sagittarius~A\textsuperscript{*}
\citep[\sgra{}, e.g.,][]{2004ApJ...616..872R}. From stellar motions near
\sgra{}, the distance from the Sun to \sgra{}, $R_0$, can be
precisely measured \citep{2009ApJ...692.1075G, 2018AA...615L..15G}. A
recent measurement using near-infrared interferometry places $R_0$ at $8.178 \pm 0.035$~kpc
\citep{2019arXiv190405721A}, or a 
     0.4\% 
uncertainty.

However, the location of \sgra{} may not be equivalent to the location
of the 
     dynamical Galactic center, the
point in three-dimensional space about which the stars in the solar
neighborhood are orbiting. This assumption, although sensible and frequently
made, has not yet been justified.

If the dynamical Galactic center is offset from \sgra{} by $100\,\pc$, only a
$1.2\%$ difference, then this induces a $\sim20\%$ error in $J_R$ for the disk-like orbits we considered (see
Section~\ref{ssec:quant}). The reason such a large error in $J_R$ can be
generated by a small error in $R_0$ can be understood from the epicyclic
approximation 
         (Equation~\eqref{eq:Ji_err_mosttime}),
which states that $\Delta J_R/J_R = 2\Delta R/A_R$. The
fractional error in $J_R$ is related to the error in $R_0$ as a fraction of
the \emph{radial amplitude} of the orbit, which is much smaller than $R_0$
($\sim1.2\,\kpc$ for \thin{} and \thick{}). This
also implies the very precise $0.3\%$ measurement of $R_0$ still translates to
a $\sim4\%$ uncertainty in $J_R$.

The assumption that the dynamical Galactic center and \sgra{} are colocated is
tested in any construction of a dynamical model where $R_0$ is a free
parameter. For example, \citet{2015ApJ...803...80K} measured $R_0$ while
modeling the dynamics of the stream Palomar 5. Many other dynamical
measurements of $R_0$ have been made (\citealt{2016ARAA..54..529B} summarize
many pre-\textit{Gaia} results), but none have yet achieved a precision
comparable to that of the distance to \sgra{}.

We did not consider in this work the effect of the \emph{angular position} of
the dynamical Galactic center being offset 
   from
\sgra{}.

\subsubsection{Galactic Orientation}
Second, one must define the angular orientation of the Galaxy. This was
defined in 1958 by the IAU subcomission 33b \citep{1960MNRAS.121..123B} by
defining the coordinates of the Galactic center in B1950 coordinates as
(17:42:26.6, -28:55:00) and the North Galactic pole as (12:49:00, +27:24:00).
These two quantities, together with $R_0$, define the orientation of the
Galactic plane. However, there is growing evidence that the stellar midplane
is tilted relative to this coordinate system \citep{2014ApJ...797...53G,
2016ARAA..54..529B}, though not the H~\textsc{ii} midplane
\citep{2019ApJ...871..145A}.

This tilt will contribute a systematic offset in $z$, with the exact magnitude
depending on the position of the observed star. For instance,
\citet{2014ApJ...797...53G} quote a $\sim0.4\degree$ tilt at $3.1\,\kpc$,
corresponding to a vertical height of $\sim22\,\pc$. This corresponds to a
$37\%$ 
error in $J_z$ for \thin{} 
    and a 5\% error for \thick{}.

\subsubsection{Solar Height}
Third, one must define the Sun's vertical distance from the Galactic midplane,
which can be determined by identifying where the stellar density and
velocities reach a maximum (effectively the median height of all disk stars).
The solar height is usually taken to be $\sim 25\,\pc$
\citep{2001ApJ...553..184C}, with a more recent measurement from \textit{Gaia}
DR2 placing it at $20.8 \pm 0.3\,\pc$ \citep{2019MNRAS.482.1417B}. Another
strategy is to use the cold gas or H~\textsc{ii} regions in the disk to define
the Galactic midplane, leading to slightly different values (by $\sim 5\,\pc$)
for the Sun's relative height \citep[e.g.,][]{2019ApJ...871..145A}. A
pre-\textit{Gaia} review of these measurements is given by
\citet{2016ARAA..54..529B}. The discrepancy between gas-based and
stellar-based determinations of the solar height is small, and thus
only likely to dominate over intrinsic midplane variations on small
scales, but will be relevant for detailed modeling of young and
kinematically cold stars. For instance, it will induce a $\sim10\%$
error in $J_z$ for an orbit with $z_{\text{max}}\sim100\,\pc$.


\subsubsection{Local Standard of Rest}
Finally, one must define the LSR, or mean velocity of stars near the Sun
relative to the Galactic center (which is defined to have zero velocity), and
the velocity of the Sun relative to the LSR. The radial ($U_{\odot}$) and
vertical ($W_{\odot}$) components are computed by taking the mean motions of
different stellar groups \citep[e.g.,][]{2012MNRAS.427..274S}. The azimuthal
component ($V_{\odot}$) is more difficult to measure, but can be modeled using
the asymmetric drift relation \citep{2008gady.book.....B}. The values of the
components of the LSR are usually taken from \citet{2010MNRAS.403.1829S}. Their uncertainties should also lead to systematic errors in the actions, as given in Equations~\eqref{eq:induced_Jphi}--\eqref{eq:induced_Jz}. For example, the
value of the circular velocity is taken to be 
         $\sim 220\,\kms$
\citep[e.g.,][]{2012ApJ...759..131B} with roughly $10\%$ uncertainty. We expect this to translate to at least a $10\%$ systematic error in $J_{\phi}$.

\subsection{Orbit Integration}\label{ssec:orbit_integrate}
We have mainly been concerned with actions, since they provide a convenient
way to quantify different types of orbits. However, all of our conclusions
         also apply to studies that 
simply rely on orbit integrations, since the two
are equivalent. For instance, computing orbital properties of open or globular
clusters \citep[e.g.,][]{2016A&A...588A.120C, 2018A&A...615A..49C,
2018A&A...616A..12G} should ideally take the midplane variation into account.
Orbit integrations of nearby systems over short time periods
\citep[e.g.,][]{2014MNRAS.445.2169M, 2018A&A...616A..37B} are unlikely to be impacted. It should
also be unimportant for halo applications, e.g., in modeling of
stellar streams \citep[e.g.,][]{2014ApJ...795...95B} or the substructure
potentially responsible for the gap in GD1 \citep{2018arXiv181103631B}.

\section{Conclusions}\label{sec:conclusion}
Determining the orbital properties of stars is important for
understanding the structure and evolution of the Galaxy. Actions have
    been argued to be 
excellent orbit labels. If the Galaxy can be 
    well
approximated as axisymmetric and 6D phase space positions can be measured
accurately and precisely enough, then the computed actions are invariant with
orbital phase. However, we have shown that the fact that the Galactic midplane
is not constant across the disk presents a significant complication to
computed actions actually being invariant. Our main conclusions are:

\begin{itemize}
\item Inaccuracy in the Galactocentric coordinate
system induces
     orbital phase 
dependence in the actions calculated from the observed
positions and velocities of stars
(Figures~\ref{fig:cartoon}~and~\ref{fig:one_orbit_wrong_ref}). Since stars'
instantaneous phase-space positions are measured without prior knowledge of
their orbital phases, this results in systematic error in the computed actions
(Figure~\ref{fig:many_orbit_wrong_ref}).

\item Inaccuracy in the midplane location most severely affects computation of
the vertical action $J_z$. A midplane offset of $\sim6\,\pc$ 
for a typical thin
disk orbit results in a $10\%$ error in $J_z$, 
       and even for a thick disk orbit a $50\,\pc$ offset will result
       in the same size error.
The fractional error is
significantly less for halo orbits.

\item The distribution of systematic errors in the actions induced by a
coordinate system offset is 
   highly
non-Gaussian. The distribution is bimodal
with \emph{neither mode at null}. As a result, error propagation of coordinate
system offsets is complex when considering actions, and is likely to
significantly deform the action-space distribution function.

\item Dynamical modeling across large regions of the disk, over which the
midplane location varies by more than the limits discussed above, is
susceptible to this type of systematic error, since the assumption that our
local Galactic midplane is the global Galactic midplane is not true a
priori. A violation of this assumption (by, e.g., intrinsic midplane
variations) leads to a systematic error in $z$ which generates the 
   large errors in actions summarized above.

\item 
    We show that such midplane variation is likely by measuring
the local galactic
midplane along the solar circle in three different high-resolution, zoom-in
simulations of Milky Way mass galaxies from the FIRE collaboration, as well as a controlled simulation of the interaction of the Milky Way with Sagittarius. We found
that the midplane varies as a function of azimuth at the solar circle by
%mm [too much detail] $\sim185$, $162$, $84\,\pc$ (middle $90\%$) for
%the three cosmological simulations we considered (\mi{}, \mf{}, and
%\mm{}, respectively) and by $\sim60$ and $\sim120\,\pc$ before and
%after the interaction in the controlled simulation. 
   60--185~pc in these simulations.
%mm [this seems redundant with the next point.] Comparison of the corresponding velocity variations with recent measurements in the Milky Way suggests that these values are consistent with expectations for midplane variations in the Milky Way itself, which have not yet been measured as a function of azimuth.


\item Assuming a vertical velocity variation of the Milky Way of
$\sim5\textup{--}10\,\kms$, consistent with recent results from
\textit{Gaia} 
and our results from the FIRE
simulations (Figure~\ref{fig:lsr_variations}), we estimated that the
corresponding midplane offsets are $\sim60\textup{--}170\,\pc$ by dimensional
analysis 
   using 
the vertical frequencies of disk-like orbits. 
    This range of values is consistent with the variations seen in the simulations.
Similar offsets are seen in three-dimensional dust maps.

\item Inaccuracies in the parameters of the currently adopted
Galactocentric coordinate system are likely important for some
applications. In particular, it is imperative to test the assumption that the
dynamical Galactic center is colocated with \sgra{}. We discuss how to
do this in Section~\ref{ssec:coord_off}.

\item This work underlines the importance of combining chemistry and dynamics.
Since chemical tagging \citep{2002ARA&A..40..487F} is not subject to the same systematic errors discussed in this paper, it should be used to confirm dynamical associations and to offset the effect of these systematic errors on the action-space distribution function.

\item While in this work we have focused on systematic errors in action
computation, all of our conclusions also extend to studies of stars that
simply rely on orbit integration, since the computation of actions and orbit
integrations are equivalent.

\end{itemize}

Our main point is that the local midplane varies between different points in
the Galaxy, and that this variation can lead to significant systematic errors
in the computation of actions under the assumption of a global axisymmetric
potential. 
Current observations from \textit{Gaia} should soon permit a measurement of the
real azimuthal dependence of the midplane location. For some applications,
such as those using actions as labels to group stars on similar orbits, using
such a measurement to shift stars to a consistent midplane height as a
function of azimuth before using a global axisymmetric approximation to the
potential may be sufficient, although this ignores the \emph{dynamical}
implications of shifts in the midplane height (which result from fluctuations
in the local density). However, for other applications, such as the study of
action diffusion,
a more extensive perturbative approach is likely to be needed. We plan to explore the
mitigation of these effects in future work.

\acknowledgments
We would like to thank Megan Bedell, Robert A. Benjamin, Tobias Buck, Elena
D'Onghia, Benoit Famaey, and Adrian Price-Whelan for helpful discussions. A.B.
would like to thank Todd Phillips for helpful discussions. This project was
developed in part at the 2019 Santa Barbara Gaia Sprint, hosted by the Kavli
Institute for Theoretical Physics (KITP) at the University of California,
Santa Barbara. This research was supported in part at KITP by the
Heising-Simons Foundation and the National Science Foundation (NSF) under
Grant No. PHY-1748958. This work used the Extreme Science and Engineering
Discovery Environment (XSEDE), which is supported by NSF Grant No.
OCI-1053575. This work uses data hosted by the Flatiron Institute's FIRE data hub. A.B. was supported in part by the Roy \& Diana Vagelos Program in
the Molecular Life Sciences and the Roy \& Diana Vagelos Challenge Award.
K.V.J.'s contributions were supported in part by the NSF under Grant No.
AST-1614743. M.-M.M.L. was partly supported by the NSF under Grant No.
AST-1815461.

\software{\texttt{matplotlib} \citep{Hunter:2007}, \texttt{scipy}
\citep{scipy:2001}, \texttt{numpy} \citep{numpy:2011}, \texttt{gala}
\citep{gala, gala:zenodo}, \texttt{astropy} \citep{astropy:2013,astropy:2018}, \texttt{tqdm} \citep{casper_o_da_costa_luis_2019_2800317}}

\appendix
\section{Orbits} \label{app:orbits}
We plot the three orbits considered throughout the work
(Table~\ref{tab:orbits}) in Figure~\ref{fig:plot_orbits}.

\begin{figure*}[htb!]
\begin{center}
\includegraphics[width=\textwidth]{fig/orbits.pdf}
\end{center}
\caption{The three orbits presented in Table~\ref{tab:orbits} and considered
throughout the work. We plot \thin{}, \thick{}, and \halo{} in the
left, center, and right columns, respectively. The upper row shows a plot of
$x$~vs.~$y$ while the lower row shows $R$~vs.~$z$.}
\label{fig:plot_orbits}
\end{figure*}

\section{$\Delta R$-$\Delta x$ Relation} \label{app:deltax}
We considered the effect on actions of an inaccuracy in the distance from the
Sun to the Galactic center, which introduces an offset in the $x$ coordinate,
$\Delta x$, of each star when converting to a Galactocentric coordinate
system. In observations of nearby stars, we have that $\Delta x \sim \Delta
R$. However, for the experiment we performed in Section~\ref{ssec:quant} we
considered observations of a star throughout its entire orbit. Therefore, we
must average $\Delta R$ over the course of the orbit. We derive this relation
now.

An offset $\Delta x$ results in an erroneous radius $R_{\text{err}}$ related by
the formula,
\beq
(x+\Delta x)^2 + y^2 = R_{\text{err}}^2\text{.}
\eeq
Keeping only terms to first order in $\Delta x$, we have that,
\beq
\begin{split}
R_{\text{err}}^2 &= R^2 - 2 R \cos{\phi} \Delta x \\
\implies \Delta R &\equiv \abs{R_{\text{err}} - R}
        = \abs{\cos{\phi}} \Delta x\text{.}
\end{split}
\eeq
Averaging over the circle, we therefore have that,
\beq
\avg{\Delta R} = \frac{2}{\pi} \Delta x\text{.}
\eeq

\section{$J_R$ and $J_{\phi}$ Distributions} \label{app:hist}

In Figure~\ref{fig:Jphi_JR_hist} we plot the distribution of $J_R$ as a
function of orbital phase induced by an offset in $x$ and $z$ and the
distribution of $J_{\phi}$ for an offset in $x$. We plot the distributions for
\thick{} (upper panels) and \thin{} (lower panels). We find that the $J_R$ distribution induced by an offset in $x$ more
closely resembles a Gaussian, while the $J_R$ distribution induced by
an offset in $z$ and the $J_{\phi}$ distribution induced by an offset
in $x$ are both similar to the $J_z$ distribution induced by an
offset in $z$ (see Figure~\ref{fig:Jz_hist}).

\begin{figure*}
\begin{center}
\includegraphics[width=7.10000594991in]{fig/schmactions_Jphi_JR_hist.pdf}
\end{center}
\caption{A histogram of the values computed for $J_R$ and $J_{\phi}$ for
\thick{} (upper panels) and \thin{} (lower panels). For $J_R$ we assume an
$x$ offset (left) and $z$ offset (center) of $100\,\pc$, while for $J_{\phi}$
we consider only an $x$ offset (right). In each panel the true value is given
by a vertical dashed line. The induced error distribution in $J_R$ for an $x$
offset more closely resembles a Gaussian centered on the null value, but not
for the other two offsets considered.}
\label{fig:Jphi_JR_hist}
\end{figure*}

\section{LSR Variations} \label{app:lsr}
We consider the variations of the LSR as a function of azimuth at the fiducial
solar circle ($R_{0} = 8.2\,\kpc$) in Figure~\ref{fig:lsr_variations}. At
each azimuth, $\phi$, we take the median velocity in cylindrical coordinates
of all stars within $200\,\pc$ of the position, following
\citet{2018arXiv180610564S}. No best-fit subtraction was performed as in
Figure~\ref{fig:midplane}.

\begin{figure*}[htb!]
\begin{center}
\includegraphics[width=\textwidth]{fig/lsr.pdf}
\end{center}
\caption{The LSR as a function of azimuth at the
fiducial solar circle ($R_0 = 8.2\,\kpc$). No best-fit subtraction is
performed here as we did in the case of the midplane
(Section~\ref{ssec:local_midplane}). Variations in $v_z$ are on the order of
$\sim5\textup{--}10\,\kms$.}
\label{fig:lsr_variations}
\end{figure*}

\bibliography{references}

\end{document}
